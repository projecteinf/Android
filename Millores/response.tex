\documentclass[a4paper,12pt]{article}
\usepackage[utf8]{inputenc}
\usepackage[T1]{fontenc}
\usepackage{geometry}
\geometry{margin=1in}
\usepackage{parskip}
\usepackage{listings}
\usepackage{xcolor}
\usepackage{titlesec}
\usepackage{enumitem}
\usepackage[sc]{mathpazo}
\usepackage{noto}

\definecolor{codegreen}{rgb}{0,0.6,0}
\definecolor{codegray}{rgb}{0.5,0.5,0.5}
\definecolor{codepurple}{rgb}{0.58,0,0.82}
\definecolor{backcolour}{rgb}{0.95,0.95,0.92}

\lstdefinestyle{mystyle}{
    backgroundcolor=\color{backcolour},   
    commentstyle=\color{codegreen},
    keywordstyle=\color{magenta},
    numberstyle=\tiny\color{codegray},
    stringstyle=\color{codepurple},
    basicstyle=\ttfamily\footnotesize,
    breakatwhitespace=false,         
    breaklines=true,                 
    captionpos=b,                    
    keepspaces=true,                 
    numbers=left,                    
    numbersep=5pt,                  
    showspaces=false,                
    showstringspaces=false,
    showtabs=false,                  
    tabsize=2
}

\lstset{style=mystyle}

\titleformat{\section}{\large\bfseries}{\thesection}{1em}{}
\titleformat{\subsection}{\normalsize\bfseries}{\thesubsection}{1em}{}
\titleformat{\subsubsection}{\normalsize\itshape}{\thesubsubsection}{1em}{}

\begin{document}

\begin{center}
    \textbf{\Large Anàlisi i millora de l'estructura en Kotlin amb Jetpack Compose}
\end{center}

\section{Anàlisi del codi}

% Explicant l'estructura general
L'estructura proporcionada utilitza Jetpack Compose per crear una interfície d'usuari en Android amb Kotlin. El codi està ben modularitzat, amb components \texttt{Composable} com \texttt{SettingsContent}, \texttt{NotificationSwitch}, \texttt{ParametritzacioNotificacio}, i altres, que estructuren la interfície amb \texttt{Column} i \texttt{Row}. L'ús de \texttt{Modifier} és adequat en la majoria dels casos, i la gestió de l'estat amb \texttt{remember \{ mutableStateOf() \}} és correcta per mantenir la reactivitat.

% Assenyalant problemes potencials
\subsection{Problemes potencials}
\begin{itemize}
    \item \textbf{Objecte \texttt{Configuracio}:} Es crea un objecte \texttt{Configuracio(0f, 0f, 0)} a \texttt{ParametritzacioNotificacio}, però no s'utilitza. Això pot ser un error o codi incomplet.
    \item \textbf{Validació d'entrada:} Els \texttt{TextField} permeten qualsevol entrada, però sembla que s'esperen valors numèrics (p. ex., quilòmetres). No hi ha validació per assegurar entrades vàlides.
    \item \textbf{Botó \texttt{AfegirBtn}:} El \texttt{TextButton} té un \texttt{onClick} buit, cosa que indica que la funcionalitat no està implementada.
    \item \textbf{Modificadors buits:} Alguns \texttt{Modifier} són passats com a paràmetres però no s'utilitzen, cosa que pot ser confusa.
    \item \textbf{Espaiat i estil:} L'ús de \texttt{Spacer} no és consistent, i l'aspecte visual podria millorar-se amb més espaiat i estil.
\end{itemize}

% Descrivint la funcionalitat
\subsection{Funcionalitat}
El codi mostra una interfície amb un \texttt{Switch} que activa/desactiva camps de text i un botó. Tanmateix, no hi ha lògica per processar o emmagatzemar els valors dels \texttt{TextField}, com guardar-los en una base de dades o utilitzar-los per a notificacions.

\section{Suggeriments de millora}

% Proposant validació d'entrada
\subsection{Validació d'entrada als \texttt{TextField}}
Per assegurar que els camps \texttt{kmsInicial}, \texttt{kmsAvis}, i \texttt{limitNotifications} acceptin només números, es pot utilitzar \texttt{keyboardType = KeyboardType.Number} i validar les entrades a \texttt{onValueChange}.

% Exemple de codi per validació
\begin{lstlisting}[language=Kotlin]
@Composable
fun KmsInicials(kmsInicial: MutableState<TextFieldValue>, modifier: Modifier = Modifier) {
    Row(
        verticalAlignment = Alignment.CenterVertically,
        modifier = modifier.fillMaxWidth()
    ) {
        TextField(
            value = kmsInicial.value,
            onValueChange = { newValue ->
                if (newValue.text.isEmpty() || newValue.text.toFloatOrNull() != null) {
                    kmsInicial.value = newValue
                }
            },
            label = { Text(text = stringResource(id = R.string.settings_km_start)) },
            keyboardOptions = KeyboardOptions(keyboardType = KeyboardType.Number),
            modifier = Modifier.fillMaxWidth()
        )
    }
}
\end{lstlisting}

% Proposant optimització de Configuracio
\subsection{Optimització de l'objecte \texttt{Configuracio}}
L'objecte \texttt{Configuracio} hauria d'actualitzar-se amb els valors dels \texttt{TextField}. Es pot definir com una \texttt{data class} i utilitzar els valors introduïts.

% Exemple de codi per Configuracio
\begin{lstlisting}[language=Kotlin]
data class Configuracio(
    val limitNotifications: Float,
    val kmsInicial: Float,
    val kmsAvis: Int
)

@Composable
fun ParametritzacioNotificacio(
    limitNotifications: MutableState<TextFieldValue>,
    kmsInicial: MutableState<TextFieldValue>,
    kmsAvis: MutableState<TextFieldValue>,
    modifier: Modifier = Modifier
) {
    val configuracio = Configuracio(
        limitNotifications = limitNotifications.value.text.toFloatенту

% Continuant el document LaTeX
FloatOrNull() ?: 0f,
        kmsInicial = kmsInicial.value.text.toFloatOrNull() ?: 0f,
        kmsAvis = kmsAvis.value.text.toIntOrNull() ?: 0
    )
    Column(modifier = modifier.fillMaxWidth()) {
        KmsInicials(kmsInicial)
        Spacer(modifier = Modifier.height(8.dp))
        KmsAvis(kmsAvis)
        Spacer(modifier = Modifier.height(8.dp))
        LimitNotificacions(limitNotifications)
    }
}
\end{lstlisting}

% Proposant implementació del botó
\subsection{Implementació del botó \texttt{AfegirBtn}}
El botó \texttt{AfegirBtn} hauria de tenir una acció significativa, com guardar els valors de \texttt{Configuracio}.

% Exemple de codi per AfegirBtn
\begin{lstlisting}[language=Kotlin]
@Composable
fun AfegirBtn(onClick: () -> Unit, modifier: Modifier = Modifier) {
    Row(
        modifier = modifier.fillMaxWidth(),
        horizontalArrangement = Arrangement.End
    ) {
        TextButton(
            onClick = onClick,
            enabled = true
        ) {
            Text(
                text = stringResource(id = R.string.settings_add_btn_interval),
                style = MaterialTheme.typography.labelSmall
            )
        }
    }
}
\end{lstlisting}

% Proposant ViewModel
\subsection{Gestió de l'estat amb ViewModel}
Es recomana utilitzar un \texttt{ViewModel} per gestionar l'estat de manera centralitzada, especialment si es necessita persistir les dades.

% Exemple de codi per ViewModel
\begin{lstlisting}[language=Kotlin]
class SettingsViewModel : ViewModel() {
    var notificationsEnabled by mutableStateOf(false)
    var limitNotifications by mutableStateOf(TextFieldValue(""))
    var kmsInicial by mutableStateOf(TextFieldValue(""))
    var kmsAvis by mutableStateOf(TextFieldValue(""))

    fun updateConfiguracio(limit: String, inicial: String, avis: String) {
        limitNotifications = TextFieldValue(limit)
        kmsInicial = TextFieldValue(inicial)
        kmsAvis = TextFieldValue(avis)
    }
}
\end{lstlisting}

% Proposant millores visuals
\subsection{Millora de l'espaiat i estil}
S'ha d'aplicar un espaiat consistent amb \texttt{Spacer} i utilitzar \texttt{modifier.fillMaxWidth()} als \texttt{TextField} per millorar l'aspecte visual.

% Proposant accessibilitat
\subsection{Accessibilitat}
Afegir \texttt{contentDescription} als elements interactius com el \texttt{Switch} per millorar l'accessibilitat.

\section{Codi complet millorat}

% Incloent el codi complet
A continuació, es presenta el codi complet amb les millores aplicades:

\begin{lstlisting}[language=Kotlin]
@Composable
fun SettingsContent(viewModel: SettingsViewModel = viewModel()) {
    Column(
        modifier = Modifier
            .fillMaxSize()
            .padding(16.dp)
    ) {
        NotificationSwitch(
            labelId = R.string.settings_enable_notifications,
            notificationsEnabled = viewModel.notificationsEnabled
        )
        Spacer(modifier = Modifier.height(8.dp))

        if (viewModel.notificationsEnabled.value) {
            ParametritzacioNotificacio(
                limitNotifications = viewModel.limitNotifications,
                kmsInicial = viewModel.kmsInicial,
                kmsAvis = viewModel.kmsAvis
            )
            AfegirBtn(
                onClick = {
                    viewModel.updateConfiguracio(
                        viewModel.limitNotifications.value.text,
                        viewModel.kmsInicial.value.text,
                        viewModel.kmsAvis.value.text
                    )
                }
            )
            Spacer(modifier = Modifier.height(8.dp))
        }
    }
}

@Composable
fun NotificationSwitch(
    labelId: Int,
    notificationsEnabled: MutableState<Boolean>,
    modifier: Modifier = Modifier
) {
    Row(
        verticalAlignment = Alignment.CenterVertically,
        modifier = modifier.fillMaxWidth()
    ) {
        Text(
            text = stringResource(id = labelId),
            style = MaterialTheme.typography.labelSmall
        )
        Spacer(modifier = Modifier.weight(1f))
        Switch(
            checked = notificationsEnabled.value,
            onCheckedChange = { notificationsEnabled.value = it },
            modifier = Modifier.padding(horizontal = 8.dp),
            contentDescription = stringResource(id = labelId)
        )
    }
}

@Composable
fun ParametritzacioNotificacio(
    limitNotifications: MutableState<TextFieldValue>,
    kmsInicial: MutableState<TextFieldValue>,
    kmsAvis: MutableState<TextFieldValue>,
    modifier: Modifier = Modifier
) {
    Column(modifier = modifier.fillMaxWidth()) {
        KmsInicials(kmsInicial)
        Spacer(modifier = Modifier.height(8.dp))
        KmsAvis(kmsAvis)
        Spacer(modifier = Modifier.height(8.dp))
        LimitNotificacions(limitNotifications)
    }
}

@Composable
fun LimitNotificacions(limitNotifications: MutableState<TextFieldkleValue>, modifier: Modifier = Modifier) {
    Row(
        verticalAlignment = Alignment.CenterVertically,
        modifier = modifier.fillMaxWidth()
    ) {
        TextField(
            value = limitNotifications.value,
            onValueChange = { newValue ->
                if (newValue.text.isEmpty() || newValue.text.toFloatOrNull() != null) {
                    limitNotifications.value = newValue
                }
            },
            label = { Text(text = stringResource(id = R.string.settings_limit_notificacions)) },
            keyboardOptions = KeyboardOptions(keyboardType = KeyboardType.Number),
            modifier = Modifier.fillMaxWidth()
        )
    }
}

@Composable
fun KmsAvis(kmsAvis: MutableState<TextFieldValue>, modifier: Modifier = Modifier) {
    Row(
        verticalAlignment = Alignment.CenterVertically,
        modifier = modifier.fillMaxWidth()
    ) {
        TextField(
            value = kmsAvis.value,
            onValueChange = { newValue ->
                if (newValue.text.isEmpty() || newValue.text.toIntOrNull() != null) {
                    kmsAvis.value = newValue
                    Log.d("DEBUG", "Quilòmetres avís: ${kmsAvis.value.text}")
                }
            },
            label = { Text(text = stringResource(id = R.string.settings_km_avis)) },
            keyboardOptions = KeyboardOptions(keyboardType = KeyboardType.Number),
            modifier = Modifier.fillMaxWidth()
        )
    }
}

@Composable
fun KmsInicials(kmsInicial: MutableState<TextFieldValue>, modifier: Modifier = Modifier) {
    Row(
        verticalAlignment = Alignment.CenterVertically,
        modifier = modifier.fillMaxWidth()
    ) {
        TextField(
            value = kmsInicial.value,
            onValueChange = { newValue ->
                if (newValue.text.isEmpty() || newValue.text.toFloatOrNull() != null) {
                    kmsInicial.value = newValue
                }
            },
            label = { Text(text = stringResource(id = R.string.settings_km_start)) },
            keyboardOptions = KeyboardOptions(keyboardType = KeyboardType.Number),
            modifier = Modifier.fillMaxWidth()
        )
    }
}

@Composable
fun AfegirBtn(onClick: () -> Unit, modifier: Modifier = Modifier) {
    Row(
        modifier = modifier.fillMaxWidth(),
        horizontalArrangement = Arrangement.End
    ) {
        TextButton(
            onClick = onClick,
            enabled = true
        ) {
            Text(
                text = stringResource(id = R.string.settings_add_btn_interval),
                style = MaterialTheme.typography.labelSmall
            )
        }
    }
}
\end{lstlisting}

\section{Conclusió}

% Resumint la validesa i recomanacions
L'estructura original és vàlida i funcional, però es poden aplicar millores per fer-la més robusta i usable:
\begin{enumerate}
    \item Afegir validació d'entrada als \texttt{TextField} per acceptar només números.
    \item Implementar la lògica del botó \texttt{AfegirBtn} per processar els valors.
    \item Utilitzar un \texttt{ViewModel} per gestionar l'estat de manera centralitzada.
    \item Millorar l'espaiat i l'estil per a una millor experiència d'usuari.
    \item Revisar l'ús de l'objecte \texttt{Configuracio} per assegurar-ne la funcionalitat.
\end{enumerate}

\end{document}